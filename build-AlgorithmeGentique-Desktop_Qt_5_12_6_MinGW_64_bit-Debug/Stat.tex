\documentclass[a4paper,10pt]{article}
\usepackage{tikz}
\usepackage{pgfplots}
\usepackage[utf8]{inputenc}
\usepackage[T1]{fontenc}
\usepackage{lmodern}
\usepackage[frenchb]{babel}
\usepackage[latin1]{inputenc}
\usepackage[francais]{babel}
\usepackage{amsmath}
\usepackage{amssymb}
\usepackage{mathrsfs}
\begin{document}
\\Récapitulatif de la configuration choisie : 
\\Taille de la population :  204
\\Type des gène :  Double\\Nombre de gènes :  2
\\Chaine évaluer :  a+b
\\Intervalle :  [0.2 , 100.1 ]
\\La génération satisfaisante :  150
\\Nombre d'itérations maximum fixé :  50
\\Type de séléction :  Séléction par rang\\Le taux de croisement :  0.05
\\Le taux de mutation :  0.7
\\Nombre d'individus à séléctionné :  190
\\Convergence : Maximisation
\\\begin{figure}[h]
\\\\\begin{tikzpicture}204
\begin{axis}[xlabel=Numéro génération ,ylabel=Moyenne scores,axis x line=bottom ,axis y line=left]
\addplot coordinates {
(0,97.1721) (1,102.835) (2,108.407) (3,111.703) (4,116.392) (5,120.608) (6,120.614) (7,121.783) (8,123.017) (9,127.832) (10,127.267) (11,128.778) (12,129.295) (13,130.466) (14,132.358) (15,133.143) (16,132.533) (17,135.125) (18,134.061) (19,135.738) (20,135.753) (21,135.292) (22,137.484) (23,138.416) (24,140.883) (25,142.105) (26,138.079) (27,142.387) (28,142.935) (29,144.069) (30,144.7) (31,142.533) (32,144.025) (33,142.935) (34,146.315) (35,146.377) (36,148.491) (37,146.505) (38,147.044) (39,146.675) (40,148.525) (41,147.354) (42,148.538) (43,149.872) (44,151.356) (45,151.351) };
\end{axis}
\end{tikzpicture}
\caption{Graphe représentant l'évolution de la moyenne des scores des individus}
\end{figure}
\begin{figure}[h]
\\\\\begin{tikzpicture}
\begin{axis}[xlabel=Numéro génération ,ylabel=Meilleur individu,axis x line=bottom ,axis y line=left]
\addplot coordinates {
(0,95.9) (1,188) (2,188) (3,188) (4,188) (5,188) (6,188) (7,188) (8,188) (9,188) (10,188) (11,188) (12,188) (13,188) (14,188) (15,188) (16,188) (17,188) (18,188) (19,188) (20,188) (21,188) (22,188) (23,188) (24,188) (25,188) (26,188) (27,188) (28,188) (29,188) (30,188) (31,188) (32,188) (33,188) (34,188) (35,192.4) (36,192.4) (37,192.4) (38,192.4) (39,192.4) (40,192.4) (41,192.4) (42,192.4) (43,192.4) (44,192.4) (45,192.4) };
\end{axis}
\end{tikzpicture}
\caption{Graphe représentant l'évolution des meilleur individus}
\end{figure}
\end{document}
