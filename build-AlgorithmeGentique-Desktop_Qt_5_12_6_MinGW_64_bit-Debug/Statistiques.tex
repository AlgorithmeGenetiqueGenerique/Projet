\documentclass[a4paper,10pt]{article}
\usepackage{tikz}
\usepackage{pgfplots}
\usepackage[utf8]{inputenc}
\usepackage[T1]{fontenc}
\usepackage{lmodern}
\usepackage[frenchb]{babel}
\usepackage[latin1]{inputenc}
\usepackage[francais]{babel}
\usepackage{amsmath}
\usepackage{amssymb}
\usepackage{mathrsfs}
\begin{document}
\\Récapitulatif de la configuration choisie : 
\\Taille de la population :  204
\\Type des gène :  Double\\Nombre de gènes :  2
\\Chaine évaluer :  a+b
\\Intervalle :  [0.2 , 100.1 ]
\\La génération satisfaisante :  150
\\Nombre d'itérations maximum fixé :  50
\\Type de séléction :  Sélétion par tournoi\\Le taux de croisement :  0.05
\\Le taux de mutation :  0.7
\\Nombre d'individus à séléctionné :  190
\\Convergence : Maximisation
\\\begin{figure}[h]
\\\\\begin{tikzpicture}204
\begin{axis}[xlabel=Numéro génération ,ylabel=Moyenne scores,axis x line=bottom ,axis y line=left]
\addplot coordinates {
(0,98.4083) (1,119.855) (2,133.547) (3,151.4) (4,158.88) };
\end{axis}
\end{tikzpicture}
\caption{Graphe représentant l'évolution de la moyenne des scores des individus}
\end{figure}
\begin{figure}[h]
\\\\\begin{tikzpicture}
\begin{axis}[xlabel=Numéro génération ,ylabel=Meilleur individu,axis x line=bottom ,axis y line=left]
\addplot coordinates {
(0,17.5) (1,147.2) (2,70.5) (3,161.9) (4,156.5) };
\end{axis}
\end{tikzpicture}
\caption{Graphe représentant l'évolution des meilleur individus}
\end{figure}
\end{document}
